\documentclass[../PaulGanssle-Thesis.tex]{subfiles}

\begin{document}
\chapter{Terminology}
% This section should be moved to an appendix.
\label{terminology}
In a developing and cross-disciplinary field such as NMR and magnetometry, there are bound to be differences in terminology. In this section, these possible ambiguities are elucidated by providing standardized definitions where they differ from the literature and citations, where the literature is unclear, or where multiple standards are used.

\section{Choice of Coordinate System}
\label{terminology.coordinates}
In the field of atomic magnetometry, particularly in the physics literature, the common practice is to call the direction of the probe beam $\vec{x}$, the direction of the measured field $\vec{y}$ and the direction of the pump beam $\vec{z}$. While this is a useful framework when the subjects under study are the rubidium spins, which are polarized along the pump beam direction, it is inconsistent with the NMR convention of defining the $\vec{z}$ axis in terms of the direction along which the spins have their initial polarization (which generally corresponds to the ``lab frame'' $\vec{z}$ axis - that is, normal to the floor). As the magnetometer described herein is used primarily for NMR studies, a sample-centered approach is used, defining the probe beam axis as $\vec{x}$, the pump beam axis as $\vec{y}$ and the sensitive direction (the direction along which the samples are polarized) as $\vec{z}$.

\section{Magnetometer Components}
\label{terminology.components}
Many components in this setup are named by analogy to a high-field instrument, or a name has been created \emph{de novo} for its use in the magnetometer.
\end{document}